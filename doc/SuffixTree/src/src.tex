\section{Описание}

Решающее действие при решении поставленной задачи - определение точного метода использования суффиксного дерева при нахождении наибольшей общей подстроки. Известно, что одно из свойств суффиксного дерева - обеспечение доступа ко всем подстрокам строки за линейное время. Таким образом, получаем, что, если объединить две исходные строки, разделив их уникальным символом, в построенном для этой строки суффиксном дереве будет в относительно явном виде, за исключением абстракций, вводимых при использовании алгоритма Укконена, существовать наибольшая общая подстрока.

\par

Основой программы-решения послужила парадигма объектно-ориентированного программирования. Были реализованы два класса - узел дерева и само дерево, связывающее узлы. Для соблюдения принципов инкапсуляции как атрибуты, так и методы, в том чиссле и конструкторы и деструкторы были описаны в качестве приватных методов, а для упрощения структуры классов в целях избежания сверхинжиниринга классов путём добавления модификаторов доступа и "геттеров" классы были сделаны "дружественными".

\pagebreak

\section{Исходный код}

\vspace{\baselineskip}

\begin{lstlisting}[language=C++]

#include <bits/stdc++.h>

class TNode 
{
	friend class TSufTree;
	
	public:
	
	~TNode() {
		for (std::pair <char, TNode *> node : this->edges) {
			delete node.second;
		}
	}
	
	private:
	
	TNode* link;
	
	int start;
	int idx = -1;
	
	std::shared_ptr<int> end;
	std::map <char, TNode* > edges;
};

class TSufTree 
{
	public:
	
	TSufTree(std::string& text, int size) 
	{
		data = text;
		firstSize = size;
		
		Init();
		SetIndex(root, 0);
	}
	
	std::pair<int, std::vector<std::string>> FindLCP()
	{
		std::vector<int> startIdx;
		Walk(root, 0, &curRes, startIdx);
		
		std::vector<std::string> result;
		std::string tmp;
		
		for (size_t i = 0; i < startIdx.size();++i) 
		{
			tmp.clear();
			
			for (int k = 0; k < curRes; k++) {
				tmp += data[k + startIdx[i]];
			}
			
			result.push_back(tmp);
		}
		
		std::sort(result.begin(), result.end());
		
		return std::make_pair(this->curRes, result);
	}
	
	~TSufTree() {
		delete root;
	}
	
	
	protected:
	
	std::string data;
	
	TNode* root = NULL;
	TNode* prevNode = NULL;
	TNode* curNode = NULL;
	
	int curEdge = -1;
	int curLen = 0;
	int curRes = 0; 
	
	int sufCnt = 0;
	
	std::shared_ptr<int> leafEnd = std::make_shared<int>(-1);
	
	int size = -1; 
	int firstSize = 0;
	
	private:
	
	TNode* AddNode(int start, std::shared_ptr<int> end) 
	{
		TNode* node = new class TNode();
		
		node->link = root;
		node->start = start;
		node->end = end;
		
		return node;
	}
	
	void Init()  
	{
		size = data.size();
		root = AddNode(-1, std::make_shared<int>(-1));
		curNode = root;
		
		for (int i = 0; i < size; i++) {
			Extend(i);
		}
	}
	
	int GetEdgeLen(TNode* n) 
	{
		if (n == root) {
			return 0;
		}
		
		return *(n->end) - (n->start) + 1;
	}
	
	
	int WalkDown(TNode* currTNode) 
	{
		if (curLen >= GetEdgeLen(currTNode)) 
		{
			curEdge += GetEdgeLen(currTNode);
			curLen -= GetEdgeLen(currTNode);
			curNode = currTNode;
			
			return 1;
		}
		
		return 0;
	}
	
	void Extend(int pos)
	{
		++*leafEnd;
		sufCnt++;
		prevNode = NULL;
		
		while (sufCnt > 0) 
		{
			if (curLen == 0) {
				curEdge = pos; 
			}
			
			if (!curNode->edges[data[curEdge]]) 
			{
				curNode->edges[data[curEdge]] = AddNode(pos, leafEnd);
				
				if (prevNode != NULL) 
				{
					prevNode->link = curNode;
					prevNode = NULL;
				}
			}
			else 
			{
				TNode* next = curNode->edges[data[curEdge]];
				
				if (WalkDown(next)) {
					continue;
				}
				
				if (data[next->start + curLen] == data[pos]) 
				{
					if (prevNode != NULL && curNode != root) 
					{
						prevNode->link = curNode;
						prevNode = NULL;
					}
					
					curLen++;
					break;
				}
				
				int splitEnd = next->start + curLen - 1;
				TNode* split = AddNode(next->start, std::make_shared<int>(splitEnd));
				curNode->edges[data[curEdge]] = split;
				
				split->edges[data[pos]] = AddNode(pos, leafEnd);
				next->start += curLen;
				split->edges[data[next->start]] = next;
				
				if (prevNode != NULL) {
					prevNode->link = split;
				}
				
				prevNode = split;
			}
			
			sufCnt--;
			
			if (curNode == root && curLen > 0) 
			{
				curLen--;
				curEdge = pos - sufCnt + 1;
			}
			else if (curNode != root) {
				curNode = curNode->link;
			}
		}
	}
	
	void SetIndex(TNode* n, int curHeight) 
	{
		if (n == NULL)  {
			return;
		}
		
		int leaf = 1;
		
		for (auto child : n->edges) 
		{
			leaf = 0;
			SetIndex(child.second, curHeight + GetEdgeLen(child.second));
		}
		
		if (leaf == 1) {
			n->idx = size - curHeight;
		}
	}
	
	int Walk(TNode* node, int curHeight, int* maxHeight, std::vector<int>& startIdx) 
	{
		if (node == NULL) {
			return 0;
		}
		
		int ret = -1;
		
		if (node->idx < 0) 
		{ 
			for (auto child : node->edges) 
			{
				ret = Walk(child.second, curHeight + GetEdgeLen(child.second), maxHeight, startIdx);
				
				if (node->idx == -1) {
					node->idx = ret;
				}
				else if (
				(node->idx == -2 && ret == -3) 
				|| (node->idx == -3 && ret == -2) 
				|| node->idx == -4 
				|| ret == -4
				) 
				{
					node->idx = -4;
					
					if (*maxHeight < curHeight) 
					{
						*maxHeight = curHeight;
						
						startIdx.clear();
						startIdx.push_back(*(node->end) - curHeight + 1);
					}
					else if (
					*maxHeight == curHeight 
					&& !startIdx.empty() 
					&& startIdx.back() != *(node->end) - curHeight + 1
					) {
						startIdx.push_back(*(node->end) - curHeight + 1);
					}
				}
			}
		}
		else if (node->idx > -1 && node->idx < firstSize) { 
			return -2;
		}
		else if (node->idx >= firstSize) { 
			return -3;
		}
		
		return node->idx;
	}
};

int main() 
{
	std::ios_base::sync_with_stdio(false);
	std::cin.tie(NULL);
	std::cout.tie(NULL);
	
	std::string first, second;
	std::cin >> first >> second;
	
	first += "#";
	second += "$";
	std::string data = first + second;
	
	TSufTree tree(data, first.size());
	std::pair<int, std::vector<std::string>> result = tree.FindLCP();
	
	std::cout << result.first << '\n';
	
	for (std::string& elem : result.second) {
		std::cout << elem << '\n';
	}
	
	return 0;
}


\end{lstlisting}

\pagebreak

