\section{Описание}

Основная суть алгоритма заключается в улучшении оценки времени работы алгоритма Бойера-Мура, в стандартном виде использующем \(6m\) сравнений, где \(m\) - длина исходного текста. Алгоритм Апостолико-Джанкарло позволяет сократить количесвто сравнений до \(2m\).

\par

Важно отметить, что эффективный алгоритм Апостолико-Джанкарло по-прежнему использует <<Правило хорошего суффикса>> и <<Правило плохого символа>>, описанные в алгоритме Бойера-Мура, нововведение заключается в таблице \(M\), хранящей информацию о количесвте совпавших символов, что позволяет не выполнять каждую фазу лгоритма Бойера-Мура <<с нуля>>.

\pagebreak

\section{Исходный код}

Ниже приведён код с реализованными <<Сильным правилом хорошего символа>> (класс TGoodSuffix), <<Сильным правилом плохого символа>> (класс TSymbolTable) и самим алгоритмом с использованием таблицы \(M\), описанными в [1]. 

\begin{lstlisting}[language=C]

class TGoodSuffix
{
	public:
	TGoodSuffix(std::vector<int>& lFunctionInput, std::vector<int>& lsFunctionInput) : 
	lFunction(lFunctionInput), 
	lsFunction(lsFunctionInput),
	size(lFunction.size()) {}
	
	uint32_t Get(uint32_t idx)
	{
		if (!idx)
		{
			if (size > 2){
				return size - lsFunction[1] - 1;
			}
			
			return 1;
		}
		
		if (idx == (uint32_t)this->size - 1){
			return 1;
		}
		
		if (lFunction[idx + 1]){
			return size - lFunction[idx + 1] - 1;  
		}
		
		return size - lsFunction[idx + 1]; 
	}
	
	private:
	std::vector<int> lFunction;
	std::vector<int> lsFunction;
	uint32_t size;
};

\end{lstlisting}

\begin{lstlisting}[language=C]

class TSymbolTable
{
	public:
	TSymbolTable(std::vector<uint32_t>& pattern){
		for (unsigned i = 0; i < pattern.size(); ++i){
			this->table[pattern[i]] = i;
		}
	}
	
	uint32_t Get(uint32_t symbol, uint32_t idx)
	{
		auto it = table.find(symbol);
		
		if (it == table.end()){
			return idx + 1;
		}
		
		return idx - it->second; 
	}
	
	private:
	std::map<uint32_t, uint32_t> table;
};

\end{lstlisting}
	
\begin{lstlisting}[language=C]
	
	std::vector<int> m(text.size(), 0);
	
	for (uint32_t j = pattern.size() - 1; j < (uint32_t)text.size();)
	{
		int h = j;
		int i = pattern.size() - 1;
		
		while (true)
		{
			if (!m[h])
			{
				if (text[h] == pattern[i] && i == 0)
				{
					std::cout << positions[j - pattern.size() + 1] << '\n';
					
					m[j] = pattern.size();
					j += succesShift.Get();
					
					break;
				}
				
				if (text[h] == pattern[i] && i > 0)
				{
					--i;
					--h;
					
					continue;
				}
				
				if (text[h] != pattern[i])
				{
					m[j] = j - h;
					
					j += std::max(goodSuffix.Get(i), symbolTable.Get(text[h], i));
					
					break;
				}
			}
			
			if (m[h] < nFunction[i])
			{
				i -= m[h];
				h -= m[h];
				
				continue;
			}
			
			if (m[h] >= nFunction[i] && nFunction[i] >= i)
			{
				std::cout << positions[j - pattern.size() + 1] << '\n';
				
				m[j] = j - h;
				j += succesShift.Get();
				
				break;
			}
			
			if (m[h] > nFunction[i] && nFunction[i] < i)
			{
				m[j] = j - h;
				j += std::max(goodSuffix.Get(i - nFunction[i]), symbolTable.Get(text[h - nFunction[i]], i - nFunction[i]));
				
				break;
			}
			
			if (m[h] == nFunction[i] && nFunction[i] > 0 && nFunction[i] < (int)i)
			{
				i -= m[h];
				h -= m[h];
				
				continue;
			}
		}
	}

	
\end{lstlisting}



\section{Консоль}
\begin{alltt}
[axr@pekarnya contest]$ g++ -Wall -o ag ApostolicoGiancarlo.cpp
[axr@pekarnya contest]$ ./ag
11 45 11 45 90
0011 45 011 0045 11 45 90    11
45 11 45 90
1, 3
1, 8
[axr@pekarnya contest]$ 
\end{alltt}
\pagebreak

