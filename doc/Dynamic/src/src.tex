\section{Описание}

"В общем случае мы можем решить задачу, в которой присутствует оптимальная подструктура, проделывая следующие три шага:

\begin{enumerate}
	\item Разбиение задачи на подхадачи меньшего размера.
	\item Нахождение оптимального решения подзадач рекурсивно, проделывая такой же трёхшаговый алгоритм.
	\item Использование полученного решения подзадач для конструирования решения исходной задачи.
\end{enumerate}

\vspace{\baselineskip}

Подзадачи решаются делением их на подзадачи ещё меньшего размера и т.д., пока не приходят к тривиальному случаю задачи, решаемой за константное время (ответ можно сказать сразу).

\vspace{\baselineskip}
В данной задаче мы будем делать проход снизу вверх, последовательно считая минимальный штраф для попадания в текущую клетку массива для всех клеток на основе уже посчитанных таким образом минимальных штрафов для клеток внизу. Очевидно, что для клеток со строки n-1 ничего считать не нужно.  

После этого их первой строки выбирается клетка с наименьшим штрафом, а затем алгорит проходит вниз по теблице, выбирая наименьшее число из трёх клеток снизу.

\pagebreak

\section{Исходный код}

\vspace{\baselineskip}

\begin{lstlisting}[language=C++]
#include <bits/stdc++.h>
#include <istream>
#include <vector>

std::istream& operator >> (std::istream& in, std::vector<std::vector<int64_t>> &data) {
	for (int64_t i = 0; i < (int64_t)data.size(); ++i){
		for (int64_t j = 0; j < (int64_t)data[0].size(); ++j){
			in >> data[i][j];
		}
	}
	
	return in;
}

void DisplayAnswer(std::vector<std::vector<int64_t>> &data)
{
	int64_t n = data.size();
	int64_t m = data[0].size(); 
	
	int64_t minValue = data[0][0];
	int64_t idx = 0;
	
	for (int64_t i = 1; i < m; ++i) {
		if (data[0][i] <= minValue) {
			minValue = data[0][i];
			idx = i;
		}
	}
	
	std::cout << minValue << '\n';
	
	int64_t i, j = idx;
	
	for (i = 0; i < n - 1; ++i) 
	{
		std::cout << '(' << i + 1 << "," << j + 1 << ") ";
		
		int64_t tmp;
		
		if (j == 0) {
			tmp = std::min(data[i + 1][j], data[i + 1][j + 1]);
		} 
		else if (j == m - 1) {
			tmp = std::min(data[i + 1][j - 1], data[i + 1][j]);
		}
		else {
			tmp = std::min({data[i + 1][j - 1], data[i + 1][j], data[i + 1][j + 1]});
		}
		
		if (tmp == data[i + 1][j + 1]) {
			j++;
		}
		else if (tmp == data[i + 1][j - 1]) {
			j--;
		}
	}
	
	std::cout << '(' << i + 1 << "," << j + 1 << ")\n";
}

int main() 
{
	int64_t n, m;
	std::cin >> n >> m;
	
	std::vector<std::vector<int64_t>> data (n, std::vector<int64_t>(m));
	
	std::cin >> data;
	
	/*
	for (int64_t i = 0; i < n; ++i) {
		for (int64_t j = 0; j < m; ++j) {
			std::cin >> data[i][j];
		}
	}
	*/
	
	for (int64_t i = n - 2; i > -1; --i) {
		for (int64_t j = 0; j < m; ++j) 
		{
			if (j == 0) {
				data[i][j] += std::min(data[i + 1][j], data[i + 1][j + 1]);
			}
			else if (j == m - 1) {
				data[i][j] += std::min(data[i + 1][j - 1], data[i + 1][j]);
			} 
			else {
				data[i][j] += std::min({data[i + 1][j - 1], data[i + 1][j], data[i + 1][j + 1]});
			}
		}
	}
	
	DisplayAnswer(data);
	
	return 0;
}

\end{lstlisting}

\pagebreak

