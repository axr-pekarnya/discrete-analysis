\CWHeader{Лабораторная работа \textnumero 1}

\textbf{Формулировка задания}: Задана матрица натуральных чисел $A$ размерности $m \times n$. Из текущей клетки можно перейти в любую из 3-х соседних, стоящих в строке с номером на единицу больше, при этом за каждый проход через клетку $(i, \, j)$ взымается штраф $A_{ij}$. Необходимо пройти из какой-нибудь клетки верхней строки до любой клетки нижней, набрав при проходе по клеткам минимальный штраф. 

\par

\textbf{Формат ввода}: Первая строка входного файла содержит в себе пару чисел $2 \leq n, \, m \leq 1000$ и , затем следует $n$ строк из m целых чисел. 

\textbf{Формат вывода}: Необходимо вывести в выходной файл на первой строке минимальный штраф, а на второй – последовательность координат из $n$ ячеек, через которые пролегает маршрут с минимальным штрафом. 

\pagebreak
