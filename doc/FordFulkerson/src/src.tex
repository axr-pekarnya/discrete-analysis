\section{Описание}

Задача решается вполне тривиально, поэтому можно ограничиться кратким описанием алгоритма. Обращаясь же к коду, важно отметить, что был реализован абстрактный класс с приватными полями, необходимыми для имплементации графа, а так же методом поиска - использовать поиск в глубину, в ширину или с рекурсивным заглублением - решается при реализации интерфейса.

\par

Алгоритм Форда-Фалкерсона использует поиск увеличивающих путей в остаточной сети для нахождения максимального потока в сети. Процесс повторяется до тех пор, пока увеличивающие пути существуют. Инициализация: Создать остаточную сеть, где пропускные способности ребер обновляются после каждого увеличивающего пути. Поиск увеличивающего пути: Используя поиск в глубину или ширину, найти путь от источника к стоку в остаточной сети. Нахождение пропускной способности пути: Найти минимальную пропускную способность ребер на увеличивающем пути. Обновление остаточной сети: Уменьшить пропускные способности по ребрам на увеличивающем пути и увеличить их обратно. Повторение: Повторять шаги 2-4, пока увеличивающие пути существуют. Максимальный поток: Суммировать пропускные способности всех увеличивающих путей. Алгоритм завершается, когда нет больше увеличивающих путей в остаточной сети.

\pagebreak

\section{Исходный код}

\vspace{\baselineskip}

\begin{lstlisting}[language=C++]

#include <bits/stdc++.h>

int INF = 2e9;

class IGraph
{
	public:
	int Size() {
		return this->data.size();
	}
	
	void SetVertex(int i, int j, int value) {
		this->data[i][j] = value;
	}
	
	int GetVertex(int i, int j) {
		return this->data[i][j];
	}
	
	bool Search(int source, int destination, std::vector<int> &parents);
	
	protected:
	std::vector<std::vector<int>> data;
};

class TGraph : public IGraph
{
	public:
	TGraph(int n) {
		this->data = std::vector<std::vector<int>>(n + 1, std::vector<int>(n + 1));
	}
	
	void SetVertex(int i, int j, int value) {
		this->data[i][j] = value;
	}
	
	int GetVertex(int i, int j) {
		return this->data[i][j];
	}
	
	bool Search(int source, int destination, std::vector<int> &parents)
	{
		std::vector<bool> visited(this->Size());
		std::queue<int> mem;
		
		mem.push(source);
		visited[source] = true;
		parents[source] = -1; 
		
		while (!mem.empty()) 
		{
			int lastVer = mem.front();
			mem.pop();
			
			for (int curVer = 0; curVer < this->Size(); ++curVer) 
			{
				if (!visited[curVer] && this->data[lastVer][curVer]) 
				{
					mem.push(curVer);
					parents[curVer] = lastVer;
					visited[curVer] = true;
					
					if (curVer == destination) {
						return true;
					}
				}
			}
		}
		
		return false;
	}
};

int64_t GetMaxFlow(TGraph &graph, int source, int destination)
{
	int64_t result = 0;
	
	TGraph resGraph = graph;
	std::vector<int> parents(graph.Size());
	
	while (resGraph.Search(source, destination, parents)) 
	{
		int curFlow = INF;
		
		for (int curVertex = destination; curVertex != source; curVertex = parents[curVertex]) {
			curFlow = std::min(curFlow, graph.GetVertex(parents[curVertex], curVertex));
		}
		
		for (int curVertex = destination; curVertex != source; curVertex = parents[curVertex]) 
		{
			resGraph.SetVertex(parents[curVertex], curVertex, resGraph.GetVertex(parents[curVertex], curVertex) - curFlow);
			resGraph.SetVertex(curVertex, parents[curVertex], resGraph.GetVertex(curVertex, parents[curVertex]) + curFlow);
		}
		
		result += curFlow;
	}
	
	return result;
}

int main(void)
{
	int n, m;
	std::cin >> n >> m;
	
	TGraph graph(n);
	
	int src, dst, value;
	
	for (int i = 0; i < m; ++i) 
	{
		std::cin >> src >> dst >> value;
		graph.SetVertex(src, dst, value);
	}
	
	std::cout << GetMaxFlow(graph, 1, n) << '\n';
	
	return 0;
}


\end{lstlisting}

\pagebreak

