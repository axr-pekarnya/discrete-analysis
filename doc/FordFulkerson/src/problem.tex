\CWHeader{Лабораторная работа \textnumero 9}

\textbf{Формулировка задания}: Задан взвешенный ориентированный граф, состоящий из $n$ вершин и $m$ ребер. Вершины пронумерованы целыми числами от 1 до $n$. Необходимо найти величину максимального потока в графе при помощи алгоритма Форда-Фалкерсона. Для достижения приемлемой производительности в алгоритме рекомендуется использовать поиск в ширину, а не в глубину. Истоком является вершина с номером $1$, стоком – вершина с номером $n$. Вес ребра равен его пропускной способности. Граф не содержит петель и кратных ребер. 

\par

\textbf{Формат ввода}: 

В первой строке заданы $1 \leq n \leq 2000$ и $1 \leq m \leq 10000$. В следующих $m$ строках записаны ребра. Каждая строка содержит три числа – номера вершин, соединенных ребром, и вес данного ребра. Вес ребра – целое число от 0 до $10^9$.  

\textbf{Формат вывода}: Необходимо вывести одно число – искомую величину максимального потока. Если пути из истока в сток не существует, данная величина равна нулю.  

\pagebreak
