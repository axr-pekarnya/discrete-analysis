\section{Описание}

Жадные алгоритмы предназначены для решения задач оптимизации.
Они обычно представляют собой последовательность шагов, на каждом из которых
предоставляется некоторое множестов выборов. В жадном алгоритме всегда делается
выбор, который кажется самым лучшим в данный момент, т.е. проводится локально
оптимальный выбор в надежде, что он приведет к оптимальному решению глобальой
задачи. Важно отметить, что жадные алгоритмы не всегда приводят к оптимальным
решениям, но во многих задачах дают нужный результат.

\par

Применим жадный алгортим к задаче выбора отрезков. Сохраним наши пары границ
отрезков в массив данных и отсортируем их по правой границе, т.е. при обращении
к данному массиву в начале будут лежать отрезки, которые заканчиваются правее
всех. Также при сохранении отрезков запомним их изначальный индекс для того
чтобы, при выводе восстановить исходный порядок.

\par

Теперь будем считать, что в точке 0, т.е. в начале покрываемого интервала, лежит
грань, котрорую мы будем двигать по мере того, как будет происходить покрытие
интервала. Эта грань будет означать самую правую точку покрытой части (в начале
это точка 0). Пока эта точка не станет $\geq M$, мы будем считать наш интервал
непокрытым.

\par

Затем будем в цикле проходиться по нашему отсортированному масиву и брать первый отрезок, левая граница которого лежит левее или равна текущей грани, а правая
граница лежит обязательно правее текущей грани. Это необходимые условия, т.к. мы
ищем минимальное число отрезков, иначе есть шанс взять отрезки, котрые не подвинут нашу грань. Если же на каком-то этапе прохода мы не нашли такого отрезка, это
говорит о том, что мы не пожем покрыть наш интервал данной выборкой отрезков.

\par 

Подходящие же отрезки будем сохранять в новый массив, который в случае порытия
интервала отсортируем по изначальному индексу и выведем в качестве ответа.
Докажем верность данного алгоритма. Благодаря тому, что на каждом шаге мы
берем отрезок с самой правой границей, то это будет гаратировать минимальность
количества выбранных отрезков. Здесь важно отметить, что жадные алгоритмы да-
ют нам лишь одно оптимальное решение, которых может быть несколько. Следо-
вательно, допустим, что мы имеем оптимальное решение задачи выбора отрезков и
на каком-то этапе мы решаем добавить наш отрезок в это решение. У нас есть два
варианта:

\begin{itemize}
	\item Этот отрезок лежит в этом решении, тогда все ок, просто перейдем к следующй
	подзадаче;
	\item Этот отрезок не лежит в этом решении, но т.к. согласно нашему алгоритму на дан-
	ном этапе мы добавляем отрезок с самой правой границей, то он сдвинет грань еще
	дальше и обязательно будет иметь пересечние со следующим отрезком в оптималь-
	ном решении, значит мы можем просто исключить отрезок лежащий в оптимальном решении на выбраннный.
\end{itemize}

\section{Исходный код}

\vspace{\baselineskip}

\begin{lstlisting}[language=C++]
#include <bits/stdc++.h>

class TSegment {
	public:
	TSegment(int left, int right, int idx){
		this->left = left;
		this->right = right;
		this->idx = idx;
	}
	TSegment() = default;
	
	int GetLeft() const {
		return this->left;
	}
	
	int GetRight() const {
		return this->right;
	}
	
	int GetIdx() const {
		return this->idx;
	}
	
	void SetLeft(int value){
		this->left = value;
	}
	
	void SetRight(int value){
		this->right = value;
	}
	
	void SetIdx(int value){
		this->idx = value;
	}
	
	private:
	int left;
	int right;
	int idx;
	
	protected:
	
};

using psb = std::pair<std::vector<TSegment>, bool>;

bool operator<(TSegment const &seg1, TSegment const &seg2){
	return seg1.GetIdx() < seg2.GetIdx();
}

bool Select(std::vector<TSegment> &data, int m, std::vector<TSegment> &result) 
{
	TSegment curSegment(-1, -1, -1);
	
	for (TSegment &segment : data) {
		if (segment.GetRight() <= result.back().GetRight()) {
			continue;
		}
		
		if (segment.GetLeft() <= result.back().GetRight()) 
		{
			if (curSegment.GetRight() < segment.GetRight()) 
			{
				curSegment = segment;
				
				if (m <= curSegment.GetRight()) 
				{
					result.push_back(curSegment);
					return true;
				}
			}
		}
		else 
		{
			if (segment.GetLeft() <= curSegment.GetRight()) 
			{
				result.push_back(curSegment);
				curSegment = segment;
				
				if (m <= curSegment.GetRight())
				{
					result.push_back(curSegment);
					return true; 
				}
			}
			else {
				return false;
			}
		}
	}
	
	return false;
}

std::istream& operator >> (std::istream& in, std::vector<TSegment> &data) {
	for (int i = 0; i < (int)data.size(); ++i)
	{
		data[i].SetIdx(i);
		
		int left, right;
		in >> left >> right;
		
		data[i].SetLeft(left);
		data[i].SetRight(right);
	}
	
	return in;
}

bool Comparator(const TSegment &a, const TSegment &b) 
{
	int aLeft = a.GetLeft();
	int aRight = a.GetRight();
	int aIdx = a.GetIdx();
	
	int bLeft = b.GetLeft();
	int bRight = b.GetRight();
	int bIdx = b.GetIdx();
	
	return std::tie(aLeft, aRight, aIdx) < std::tie(bLeft, bRight, bIdx);
}

int main() 
{
	int n, m;
	std::cin >> n;
	
	std::vector<TSegment> data(n);
	
	std::cin >> data;
	std::cin >> m;
	
	std::sort(data.begin(), data.end(), Comparator);
	
	std::vector<TSegment> resData = {{-1, 0, -1}};
	bool found = Select(data, m, resData);
	
	std::sort(resData.begin(), resData.end());
	
	if (found) {
		std::cout << resData.size() - 1 << '\n';
		
		for (int i = 1; i < (int)resData.size(); ++i){
			std::cout << resData[i].GetLeft() << ' ' << resData[i].GetRight() << '\n';
		}
	} 
	else {
		std::cout << "0\n";
	}
	
	return 0;
}

\end{lstlisting}

\pagebreak

